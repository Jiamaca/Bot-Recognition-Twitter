\chapter{Fundamentação}\label{chp:FUNDAMENTACAO}
\section{Twitter}
O Twitter foi escolhido como nossa rede social em detrimento a outras como Facebook ou Instagram por possuir uma política mais liberal quanto ao acesso aos dados do usuário. Mesmo no acesso gratuito, a API do Twitter permite obter muitos dados de um determinado usuário, possibilitando a coleta de metadados de um perfil, os seus tweets e também os dados de quem ele se relaciona.  

\section{Problema de Classificação}
Problemas de classificação são comuns no nosso dia a dia. Mesmo que de forma inconsciente, classificamos coisas diaramente. Era esperado então que, em algum momento, com a evolução dos computadores, pudéssemos delegar esse tipo de problema para as máquinas.

De forma geral, esse tipo de problema consiste em decidir se um objeto pertence a uma determinada classe a partir de um conjunto de dados ou informações sobre esse determinado objeto. Geralmente, esse processo de aprender a classificar é supervisionado. Ou seja, sabemos a resposta correta mesmo antes do processo de aprendizado. 

\section{Bots}

\section{Revisão da Literatura}

De forma geral, as abordagens que consistem em aprendizado de máquina encontradas na literatura podem ser divididas de acordo com o tipo das features utilizadas, que podem ser: 
\begin{enumerate}
    \item Dados do Usuário
    \item Tweets de um Usuário
    \item Dados da Vizinhança de um Usuário
\end{enumerate}

\subsection{Análise dos dados do Usuário}
Na API do Twitter temos acesso a diversos dados relacionados a um usuário. Contudo, nem todos possuem valor para as análises. Portanto, somentes os dados relevantes são mostrados na Tabela 1.



\begin{table}[]
\begin{tabular}{|l|l|l|}
\hline
\textbf{Dados do Usuário}& \textbf{Tipo} & \textbf{Descrição} \\ \hline
Name & Texto &Representa o nome do usuário \\ \hline
Screen Name  & Texto  &  Representa o @ do usuário\\ \hline
Location & Texto   & Localização definida pelo próprio usuário                             \\ \hline
Url  & Texto   & URL atrelada ao perfil                            \\ \hline
Protected & Booleano & Identifica se um perfil é protegido ou não \\ \hline
Verified & Booleano& Identifica se um perfil é verificado ou não \\ \hline
Followers Count & Inteiro & Número de seguidores \\ \hline
Friends Count & Inteiro & Número de contas seguidas \\ \hline
Listed Count & Inteiro & Número de listas em que um usuário é membro \\ \hline
Favorite Counts & Inteiro & Número de publicações curtidas \\ \hline
Statuses Count & Inteiro & Número total de Tweets/Retweets \\ \hline
Profile Image Url & Texto & Link para a imagem de perfil da conta \\ \hline
Default Profile & Booleano & Identifica se o usuário alterou tema ou background do perfil\\ \hline
Created At & Texto & Data de criação da conta\\ \hline
Default Profile Image & Booleano & Identifica se a imagem de perfil é a padrão ou não \\ \hline
Geo Enabled & Booleano & Identifica se o usuário permite ser geolocalizado\\ \hline
\end{tabular}
\end{table}
\subsection{Análise textual dos Tweets}

\subsection{Análises utilizando Grafos}

\subsection{Análises Híbridas}