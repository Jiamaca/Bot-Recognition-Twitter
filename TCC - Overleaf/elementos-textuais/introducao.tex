\chapter{Introdução}\label{chp:INTRODUCAO}

As redes sociais estão cada vez mais presentes no nosso dia, estima-se que hoje, 4.2 bilhões de pessoas estão nas redes socias, ou seja, um pouco mais da metade da população mundial. 
Evidentemente, a maior parte desses usuários utiliza as redes sociais para fins comunicativos ou para entretenimento. Contudo, esse ambiente social extremamente populoso tem sido um solo fértil para o surgimento perfis automatizados(Bots) que objetivam propagar fraudes, notícias falsas e diversas outras irregularidades. 
Ao longo dos anos diversas possíveis soluções foram apresentadas para mitigar o crescimento de tais bots malignos, e com isso, iniciou-se uma "guerra entre gatos e ratos", onde cada avanço rumo a identificação desses perfis proporcionou a sofistificação dos mesmos, que atualmente, buscam até mesmo imitar o comportamento humano para evitar um possível banimento.


\section{Objetivo}

Portanto, nesse trabalho, iremos explicar as principais abordagens utilizadas previamente na literatura para identificar bots no Twitter e também apresentaremos um modelo de inteligência artificial supervisionada que objetiva identificar se um perfil é bot ou humano. Note que não faremos distinção entre bots malignos, benignos ou neutros. Visto que nosso modelo utiliza apenas os dados do usuário e das outras contas nas quais ele se relaciona. 

\section{Organização do Trabalho}

Esse documento está organizado da seguinte forma: no Capítulo 2, definiremos alguns fundamentos necessários para o entendimento do nosso domínio e explicaremos as diferentes abordagens já existentes na literatura explicando algums de suas vantagens e desvantagens. No Capítulo 3, apresentaremos o nosso modelo e analisaremos sua eficácia. Por último no Capítulo 4, faremos a conclusão com base nos resultados obtidos e discutiremos os obstáculos encontrados ao longo do processo de desenvolvimento do modelo.