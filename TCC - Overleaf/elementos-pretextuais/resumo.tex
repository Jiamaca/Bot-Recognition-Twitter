O crescimento estrondoso das redes sociais traz consigo diversos avanços para a comunicação e a conexão entre pessoas ao redor do globo. No entanto, nota-se que, atrelado a tal fato, temos visto também o aumento de perfis falsos automatizados cujo objetivo, na maioria das vezes, é manipular a opinião dos usuários ou espalhar golpes. Logo, objetivo desse trabalho é revisar o estado da arte das técnicas de identificação de Bots no Twitter e propor um modelo que não utilize informações textuais para tal analíse. Utilizaremos somente os dados do usuário e as informações de sua vizinhança(seguidores ou amigos).